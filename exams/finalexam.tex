\documentclass[12pt,letterpaper,addpoints]{exam}
\RequirePackage{amssymb, amsfonts, amsmath, latexsym, verbatim, xspace, setspace, mathrsfs}
\usepackage{amsmath,amsthm,amssymb,amsfonts, 
hyperref, color, graphicx, enumitem,multicol}
\RequirePackage{tikz, pgflibraryplotmarks}

\usepackage{geometry}
\usepackage{graphicx}
\usepackage{pstricks-add}
\usepackage{auto-pst-pdf}
\usepackage{tkz-euclide}
\usetkzobj{all}
\usepackage{sansmath}
\usepackage{etoolbox}
\pretocmd{\pshlabel}{\color{RoyalBlue4}\sansmath}{}{}
\pretocmd{\psvlabel}{\color{RoyalBlue4}\sansmath}{}{}{}%
\newcommand{\N}{\mathbb{N}}
\newcommand{\Z}{\mathbb{Z}}
\newcommand{\Q}{\mathbb{Q}}
 \everymath{\displaystyle}
\newenvironment{problem}[2][Problem:]{\begin{trivlist}
\item[\hskip \labelsep {\bfseries #1}\hskip \labelsep {\bfseries #2}]}{\end{trivlist}}

\newenvironment{claim}[2][Claim:]{\begin{trivlist}
\item[\hskip \labelsep {\bfseries #1}\hskip \labelsep {\bfseries #2}]}{\end{trivlist}}

\newenvironment{defn}[2][Definition:]{\begin{trivlist}
\item[\hskip \labelsep {\bfseries #1}\hskip \labelsep {\bfseries #2}]}{\end{trivlist}}

% Here's where you edit the Class, Exam, Date, etc.
\newcommand{\class}{Calculus 1 for Social Science}
\newcommand{\term}{Summer 2019}
\newcommand{\examnum}{}
\newcommand{\examdate}{Final Exam}
\newcommand{\timelimit}{180 min}

% For an exam, single spacing is most appropriate
\singlespacing
% \onehalfspacing
% \doublespacing

% For an exam, we generally want to turn off paragraph indentation
\parindent 0ex

\begin{document} 

% These commands set up the running header on the top of the exam pages
\pagestyle{head}
\firstpageheader{Name:}{Final Exam}{}
\runningheader{\class}{\examnum\ - Page \thepage\ of \numpages}{\examdate}
\runningheadrule

\begin{flushright}
\begin{tabular}{p{2.8in} r l}
\textbf{\class} & \textbf{Name:} & \makebox[2in]{\hrulefill}\\
\textbf{\term} &&\textbf{\examnum}\\
\textbf{\examdate} &&
\textbf{Time Limit: \timelimit}  \\ 
\end{tabular}\\
\end{flushright}
\rule[1ex]{\textwidth}{.1pt}




\begin{minipage}[t]{\textwidth}
\vspace{0pt}
\begin{itemize}

\item \textbf{DO NOT open the exam booklet until you are told to begin. You should write your name and section number at the top and read the instructions.}

\vfill

\item Organize your work, in a reasonably neat and coherent way, in
the space provided. If you wish for something to not be graded, please strike it out neatly. I will grade only work on the exam paper, unless you clearly indicate your desire for me to grade work on additional pages.

\item You may use any results from class, homework or the text, but you must cite the result you are using. You must prove everything else.

\item You needn't spend your time rewriting definitions or axioms on the exam.

\item Show all of your work.  You may not receive full credit for correct answers if supporting work is not demonstrated.
\end{itemize}


\end{minipage}
\hfill

\begin{minipage}[t]{2.3in}
\vspace{3pt}
\cellwidth{1.5em}
\gradetablestretch{1.5}
%Uncomment this line to make the table display 100 as the total no matter what. This is good for tests with an ommit question.
%\settabletotalpoints{100}
\vqword{Problem}
\addpoints % required here by exam.cls, even though questions haven't started yet.	
\gradetable[h]%[pages]  % Use [pages] to have grading table by page instead of question

\end{minipage}


\newpage
\addpoints
\begin{questions}

\question Evaluate the following limits.  If it doesn't exist, explain why.

\begin{parts}
  \part[4] $\displaystyle\lim_{x\rightarrow 4} \frac{x^2-16}{x-4}$
  \begin{solution}
  8
  \end{solution}
  
  \vfill

\part[4] $\displaystyle\lim_{x\rightarrow\infty} \frac{5x^4-3x+2}{7x^4+3x^2-1} $
\begin{solution}
$\displaystyle\frac{5}{7}$
\end{solution}

\vfill

\part[4] $\displaystyle\lim_{\displaystyle x\rightarrow0^+}\frac{1}{1+2^{-1/x}}$
\begin{solution}
$1$
\end{solution}

\vfill
\end{parts}

\newpage
\addpoints


\question[5] Use the definition of the derivative to determine the derivative of $f(x) = \frac{-1}{x}$
\begin{solution}
$f'(x) = \frac{2}{x^3}$
\end{solution}

\vfill
\question[5] Find the values of $x$ for which the function $f(x)$ is discontinuous. $$\displaystyle f(x)=\begin{cases} -2x+1 & \text{if}\;\; x<0 \\ x^2+1 & \text{if} \;\; 0\leq x \leq 1  \\ \sqrt{x+1} & \text{if} \;\; 1<x< \infty \end{cases}$$

\begin{solution}
$x=1,5$
\end{solution}

%% \question[5] Calculate $\displaystyle \lim_{x\rightarrow\infty} \frac{3x^3-x +1}{7x^3+88x^2-7}$
\vfill

\newpage
\addpoints

\question For the function $f(x) = \displaystyle x^3+2x^2+x$
\begin{parts}
  \part[3] Find the critical numbers.
  \begin{solution}
    $x=0,2$
  \end{solution}
  
  \vfill
  \part[3] Find the intervals where $f(x)$ is increasing and decreasing and identify any relative maximum and minimum values.
  \begin{solution}
    increasing on $(-\infty,0)$ and $(2,\infty)$.  decreasing on $(0,2)$.  Relative maximum at $x=0$, relative minimum at $x=2$
  \end{solution}
\vfill
    \part[5] Find the regions of concavity.
  \begin{solution}
    Concave up on $(\sqrt[3]{2},\infty)$, Concave down on $(-\infty,\sqrt[2]{2})$
  \end{solution}
  

\vfill

  \vfill
  \part[2] Find the $x$-coordinate of any inflection points.
  \begin{solution}
    $x=\sqrt[3]{2}$
  \end{solution}
  
  \vfill
\newpage
\addpoints

\part[5] Plot the function.

\psset{unit=8mm, ticks=none, xlabelsep=1pt, ylabelsep=1pt, arrowinset=0.12}%, l

\psset{gridwidth=.7pt, subgriddiv=1,gridlabels=0pt}
\begin{pspicture*}(-8,-8.5)(8,8)
\psgrid(-8,-8)(8,8.5)
%\psaxes[labelFontSize=\scriptstyle, linecolor=SteelBlue]{<->}(0,0)(-8,-8)(8,8)[\textsf{X}\rule{0pt}{2.25ex} ,-120][\textsf{Y} , -150]
\psset{linecolor=DodgerBlue4, tickcolor=white, subtickcolor=DodgerBlue4, gridlabelcolor=DodgerBlue4, ,linewidth = 1.2pt}%
\end{pspicture*}

\end{parts}

\question[5]  Find the equation of the tangent line to the curve $y^2=x^3(2-x)$ at the point $(x,y)=(-1,2)$.
\begin{solution}
$y=5x+3$
\end{solution}

\vfill


\newpage
\addpoints
\question Find $\displaystyle\frac{dy}{dx}$ for the following equations.
\begin{parts}
  \part[5] $y = \tan(x^3+1)$
  \begin{solution}
    $y'=3x^2(\tan^2(x^3+1)+1)$
  \end{solution}
  
\vfill
  \part[5] $y = \displaystyle \arcsin\left(e^{2x}\right)$
  \begin{solution}
    $y'=\displaystyle\frac{2e^{2x}}{\sqrt{-e^{4x}+1}}$
  \end{solution}
  \vfill
  \part[5] $y = (x^2+1)\arctan(\sqrt{x})$
  \begin{solution}
    $2x\arctan(\sqrt{x}) + \frac{(x^2+1)}{2\sqrt{x}(x+1)}$
  \end{solution}
  
\vfill
\end{parts}


\question[5] Use Logorithmic differentiation to find the derivative of $y=\displaystyle \frac{\displaystyle(x^3+1)^4\sin^2(x)}{\displaystyle\sqrt[3]{x}}$.  DO NOT SIMPLIFY YOUR ANSWER.
\begin{solution}
$\displaystyle\left(\frac{12x^2}{x^3+1}+2\cot(x)-\frac{1}{3x} \right)\left(\frac{\displaystyle(x^3+1)^4\sin^2(x)}{\displaystyle\sqrt[3]{x}} \right)$
\end{solution}

\vfill

\newpage
\addpoints


\question A company's cost function and demand function are given by
$$
C(x) = 3800 + 5x-\frac{x^2}{1000} \;\;\text{and}\;\; p(x) = 50-\frac{x}{100} \;\; \text{for}\;\;0\leq x\leq 1000.
$$

\begin{parts}
  \part[4] Compute $C'(300)$ and give an interpretation of the results.
  \begin{solution}
    $\frac{22}{5}$, this is the cost of producing the 301 unit.
  \end{solution}
  
  \vfill
  \part[2] Find the revenue function $R(x)$.
  \begin{solution}
    $R(x) = 50x-\frac{x^2}{100}$
  \end{solution}
  
  %% \vfill
  
  %% \part[3] Compute $R'(300)$ and give an interpretation of the results.
  %% \begin{solution}
  %%   $50-\frac{300}{50}$, this is the revenue obtained from the 301 unit.
  %% \end{solution}
  
  \vfill
  \part[4] Find the profit function $P(x)$ and the demand level which maximizes profit.
  \begin{solution}
    $P(x) = \frac{-9x^2}{1000}+45x-3800$, and $x=2500$
  \end{solution}
  
\end{parts}
\vfill

\newpage
\addpoints

\question[5] Find the horizontal (if any) and vertical (if any) asymptote(s) of the function $$f(x) = \frac{2x^2-6x}{x^2-9}$$
\begin{solution}
vertical asymptote $x=- 3$ and horizontal asymptote $y=2$
\end{solution}

\vfill

\question[10]
A box with an open top and a square base is to have a volume of 32000 cm$^3$.
Find dimensions of the box which will minimize the surface area of the box.
\begin{solution}
40 cm by 40 cm by 20 cm
\end{solution}
\vfill

\newpage
\addpoints

\question[10] A ladder 10ft long rests against a vertical wall.  If the bottom of the ladder slides away from the wall at a rate of 2 ft/s, how fast is the top of the ladder sliding down the wall when the bottom of the ladder is 6ft from the wall?

\begin{solution}
1.5 ft/s
  \end{solution}
%% \question[5] Factor completely $7x^3 - 42x^2y+63xy^2$.

%% \question[5] Factor completely $6x^2 + 13x -5$.

%% \question[5] Solve the system of linear equations using the method of your choosing $\begin{cases} 2x+y = 3 \\ 3x-4y=1 \end{cases}$

%% \question[5] Find the equation of the line perpendicular to the line $y=\frac{1}{3}x+1$ going through the point $(4,2)$.

%% \question[5] Graph the line $2x - y = 2$

%%   \psset{unit=6mm, ticks=none, xlabelsep=1pt, ylabelsep=1pt, arrowinset=0.12}%, l
%% \psset{gridwidth=0.3pt, subgriddiv=1,gridlabels=0pt}
%% \begin{pspicture*}(-8,-8.5)(8,8)
%% \psgrid(-8,-8)(8,8.5)
%% \psaxes[labelFontSize=\scriptstyle, linecolor=SteelBlue]{<->}(0,0)(-8,-8)(8,8)[\textsf{X}\rule{0pt}{2.25ex} ,-120][\textsf{Y} , -150]
%% \psset{linecolor=DodgerBlue4, tickcolor=white, subtickcolor=DodgerBlue4, gridlabelcolor=DodgerBlue4, ,linewidth = 1.2pt}%
%% \end{pspicture*}

%% \question[5] Find the value of $x$ in the following triangle.

%% \begin{tikzpicture}[thick]
%% \draw(0,0) -- (90:2cm) node[midway,left]{$x-3$} -- (0:4cm) node[midway,above right]{$7$} -- (0,0) node[midway,below]{$5$};
%% \end{tikzpicture}

%% \question[5] Rationalize the denominator and simplify. $\displaystyle \frac{\sqrt{2}}{3-\sqrt{2}}$

%% \question[5] Simplify and write with only positive exponents. $\displaystyle \frac{(x^4)^{-2}(yx^{-1})^2}{(2xy)^{-1}}$

%% \question[5] Perform the operations and write the answer in lowest terms.  Assume all numerators and enominators are nonzero. $\displaystyle\frac{(3x+2)(2x-3)}{3x+17x+10}\div\frac{(x-3)(2x-3)}{x(x+5)}$

%% \question[5] for the quadratic function $f(x) = -2x^2 + 8x - 5$
%% \begin{description}
%% \item[a.] Find the $x$ and $y$ intercepts
%% \item[b.] find the vertex.
%% \end{description}

%% \question[5] The sum of 3 consecutive even numbers is 132.  Find the numbers.

%% \question[5] Simplify $(2\sqrt{5}+\sqrt{3})\cdot(3\sqrt{5}-5\sqrt{3})$

%% %\question[5] Simplify $(3x^2y - 5xy^2 - xy + 5) - (5x^2y + 3xy^2 +x +5)$

%% \question[5] Solve $\log_{\sqrt{2}} (1-3x) = 2$

%% \question[5] Simplify and write answer in lowest terms $\displaystyle\frac{\displaystyle\left(\frac{2x+1}{4}\right)}{\displaystyle \left( 5-\frac{3}{x}\right)}$

%% \question[5] Dennis mowed his next door neighbour's lawn for a handful of dimes and nickels, 80 coins in all.  Upon completing the job he counted out the coins and it came to \$6.60.  How many of each coin does he have?

%% \question[5] Solve $49x^2-196 = 0$

%% \question[5] Solve $\sqrt{2x+1} - 3 = x -4$

%% \question[5] Perform the operations and write the answer in lowest terms. $$\displaystyle \frac{5x}{x^2-1}-\frac{4}{x^2+x-2}$$

%% \question[5]  A rectangle has sides of length $x+3$ and $4x-1$ and has total area of 270.  Find the length of each side.


%% \question[5] For the triangle below
%% \begin{description}

%% \item[a.] find $\sin\beta$, $\cos\beta$ and $\tan\beta$
%% \item[b.] calculate $\sin\beta$ + $\cos\alpha$
%% \item[c.] calculate $\sin^2\beta + \cos^2\beta$.
%% \end{description}

%% \begin{tikzpicture}[thick]
%% \coordinate (O) at (0,0);
%% \coordinate (A) at (4,0);
%% \coordinate (B) at (0,2);
%% \draw (O)--(A)--(B)--cycle;

%% %\tkzLabelSegment[below=2pt](O,A){\textit{adjacent leg}}
%% \tkzLabelSegment[left=2pt](O,B){2}
%% \tkzLabelSegment[above right=2pt](A,B){8}

%% \tkzMarkRightAngle[size=0.5,opacity=.4](A,O,B)% square angle here
%% %\tkzLabelAngle[pos = 0.35](A,O,B){$\gamma$}

%% \tkzMarkAngle[size=.99cm,%
%% opacity=.4](B,A,O)
%% \tkzLabelAngle[pos = 0.75](B,A,O){$\alpha$}

%% \tkzMarkAngle[size=0.9cm,%
%% opacity=.4](O,B,A)
%% \tkzLabelAngle[pos = 0.5](O,B,A){$\beta$}

%% \end{tikzpicture}


%% \section*{answers}
%% \begin{multicols}{2}
  
%% \begin{enumerate}
%% \item $7x(x-3y)^2)$
%% \item $(2x+5)(3x-1)$
%% \item $(x,y)=(\frac{13}{11},\frac{7}{11})$
%% \item $y=-3x+14$
%% \item line with y intercept $y=-2$ and $x$ intercept $x=1$
%% \item $3+2\sqrt{6}$
%% \item $\frac{2+3\sqrt{2}}{7}$
%% \item $\frac{2y^3}{x^9}$
%% \item $\frac{x}{x-3}$
%% \item a. $y=-5$, $x=2\pm\frac{\sqrt{6}}{2}$ b. $(x,y)=(2,3)$
%% \item 42,44,46
%% \item $15 -7\sqrt(15)$
%% \item $x=-\frac{1}{3}$
%% \item $\frac{x(2x+1)}{4(5x-3)}$
%% \item 52 dimes and 28 nickels
%% \item $\pm 2$
%% \item $2+\sqrt(2)$
%% \item $\frac{5x^2+6x-4}{x^3+2x-x-2}$
%% \item 10,27
%%   \item a. $\sin(\beta)=\sqrt{15}/4$,$\cos(\beta)=1/4$, $\tan(\beta)=\sqrt{15}$ b. $\sqrt{15}/2$, c. 1
  
%% \end{enumerate}

%% \end{multicols}
    

\end{questions}
\end{document}
