\documentclass[11pt,letterpaper]{exam}
\RequirePackage{amssymb, amsfonts, amsmath, latexsym, verbatim, xspace, setspace, mathrsfs}
\usepackage{amsmath,amsthm,amssymb,amsfonts, 
hyperref, color, graphicx, enumitem}
\RequirePackage{tikz, pgflibraryplotmarks}
\usepackage[margin=0.5in]{geometry}
\usepackage{graphicx}
\usepackage{tabularx}
\newcommand{\N}{\mathbb{N}}
\newcommand{\Z}{\mathbb{Z}}
\newcommand{\Q}{\mathbb{Q}}
\addtolength{\topmargin}{0.5in}
\newenvironment{problem}[2][Problem:]{\begin{trivlist}
\item[\hskip \labelsep {\bfseries #1}\hskip \labelsep {\bfseries #2}]}{\end{trivlist}}

\newenvironment{claim}[2][Claim:]{\begin{trivlist}
\item[\hskip \labelsep {\bfseries #1}\hskip \labelsep {\bfseries #2}]}{\end{trivlist}}

\newenvironment{defn}[2][Definition:]{\begin{trivlist}
\item[\hskip \labelsep {\bfseries #1}\hskip \labelsep {\bfseries #2}]}{\end{trivlist}}

% Here's where you edit the Class, Exam, Date, etc.
\newcommand{\class}{Calculus 1 (social sciences)}
\newcommand{\term}{Winter 2019}
\newcommand{\examnum}{}
\newcommand{\examdate}{Quiz 8}
\newcommand{\timelimit}{}

% For an exam, single spacing is most appropriate
\singlespacing
% \onehalfspacing
% \doublespacing

% For an exam, we generally want to turn off paragraph indentation
\parindent 0ex

\begin{document} 

% These commands set up the running header on the top of the exam pages
\pagestyle{head}
\firstpageheader{Name:}{Quiz 8}{}
\runningheader{\class}{\examnum\ - Page \thepage\ of \numpages}{\examdate}
\runningheadrule

%\begin{flushright}
%\begin{tabular}{p{2.8in} r l}
%\textbf{\class} & \textbf{Name:} & \makebox[2in]{\hrulefill}\\
%\textbf{\term} &&\textbf{\examnum}\\
%\textbf{\examdate} &&
%\textbf{Time Limit:  \timelimit}  \\ 
%\end{tabular}\\
%\end{flushright}
%\rule[1ex]{\textwidth}{.1pt}




%\begin{minipage}[t]{3.7in}
%\vspace{0pt}
%\begin{itemize}

%% \item \textbf{DO NOT open the exam booklet until you are told to begin. You should write your name and section number at the top and read the instructions.}

%% \vfill

%% \item Organize your work, in a reasonably neat and coherent way, in
%% the space provided. If you wish for something to not be graded, please strike it out neatly. I will grade only work on the exam paper, unless you clearly indicate your desire for me to grade work on additional pages.

%% \item You may use any results from class, homework or the text, but you must cite the result you are using. You must prove everything else.

%% \item You needn't spend your time rewriting definitions or axioms on the exam.

%% \end{itemize}


%% \end{minipage}
%% \hfill
%% \begin{minipage}[t]{2.3in}
%% \vspace{0pt}
%% %\cellwidth{3em}
%% \gradetablestretch{2}
%% %Uncomment this line to make the table display 100 as the total no matter what. This is good for tests with an ommit question.
%% %\settabletotalpoints{100}
%% \vqword{Problem}
%% \addpoints % required here by exam.cls, even though questions haven't started yet.	
%% \gradetable[v]%[pages]  % Use [pages] to have grading table by page instead of question

%% \end{minipage}

%% \begin{itemize}


%% \item You may use the text, my class notes and/or any notes and study guides you have created. You may use a calculator. You may not use a cell phone or computer.


%% \item When you have completed your test, hand it to me and go have a great weekend!

%% \item There is a single bonus problem at the end of the test. It would be best to work first on the main test as this problem is only worth 5 points and will be graded strictly.

%% \end{itemize}

%%\newpage
\begin{questions}
\question[15pts] Plot the function $f(x) = \displaystyle \frac{2}{3}x^3-2x-6x+1$. Make sure you indicate all extrema, inflection points, $x$ and $y$ intercepts, any asymptotes which may exist.  

%% \vspace{.5in}

%% \textbf{Question 3} (10 points)

%% The demand function for a product is $p = -0.04x+800$ where $x$ is the quantity demanded and $p$ is the unit price.
%% \begin{description}
%% \item[a.] find the revenue function.
%% \item[b.] find the marginal revenue function.
%%   \item[c.] Evaluate $R'(5000)$ and interpret your result.
%% \end{description}


%% \begin{questions}
%% \addpoints
%% \question[5] Using the definition of the derivative to evaluate the derivative of $f(x) = 8x^2 +1$.

%% \vfill

%% \question[5]
%% Find the derivate of $f(x) = (x^2+x+1)(\sqrt{x} - x)$
%% \vfill

%% \newpage

%% \question[10]
%% for the function $f(x) = 3x^2$
%% \begin{enumerate}
%% \item Find $f'(x)$
%%   \vfill
%%   \item find the equation of the tangent line to $f(x)$ at the point $x=3$
%% \end{enumerate}
%% \vfill
\end{questions}
\end{document}
