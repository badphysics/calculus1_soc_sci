\documentclass[12pt,letterpaper]{exam}
\RequirePackage{amssymb, amsfonts, amsmath, latexsym, verbatim, xspace, setspace, mathrsfs}
\usepackage{amsmath,amsthm,amssymb,amsfonts, 
hyperref, color, graphicx, enumitem}
\RequirePackage{tikz, pgflibraryplotmarks}
\usepackage{geometry}
\usepackage{graphicx}

\newcommand{\N}{\mathbb{N}}
\newcommand{\Z}{\mathbb{Z}}
\newcommand{\Q}{\mathbb{Q}}
 \everymath{\displaystyle}
\newenvironment{problem}[2][Problem:]{\begin{trivlist}
\item[\hskip \labelsep {\bfseries #1}\hskip \labelsep {\bfseries #2}]}{\end{trivlist}}

\newenvironment{claim}[2][Claim:]{\begin{trivlist}
\item[\hskip \labelsep {\bfseries #1}\hskip \labelsep {\bfseries #2}]}{\end{trivlist}}

\newenvironment{defn}[2][Definition:]{\begin{trivlist}
\item[\hskip \labelsep {\bfseries #1}\hskip \labelsep {\bfseries #2}]}{\end{trivlist}}

% Here's where you edit the Class, Exam, Date, etc.
\newcommand{\class}{Calculus 1 for Social Sciences}
\newcommand{\term}{Summer 2019}
\newcommand{\examnum}{}
\newcommand{\examdate}{Exam 1}
\newcommand{\timelimit}{90 min}

% For an exam, single spacing is most appropriate
\singlespacing
% \onehalfspacing
% \doublespacing

% For an exam, we generally want to turn off paragraph indentation
\parindent 0ex

\begin{document} 

% These commands set up the running header on the top of the exam pages
\pagestyle{head}
\firstpageheader{Name:}{Exam 1}{}
\runningheader{\class}{\examnum\ - Page \thepage\ of \numpages}{\examdate}
\runningheadrule

\begin{flushright}
\begin{tabular}{p{2.8in} r l}
\textbf{\class} & \textbf{Name:} & \makebox[2in]{\hrulefill}\\
\textbf{\term} &&\textbf{\examnum}\\
\textbf{\examdate} &&
\textbf{Time Limit: \timelimit}  \\ 
\end{tabular}\\
\end{flushright}
\rule[1ex]{\textwidth}{.1pt}




\begin{minipage}[t]{3.7in}
\vspace{0pt}
\begin{itemize}

\item \textbf{DO NOT open the exam booklet until you are told to begin. You should write your name and section number at the top and read the instructions.}

\vfill

\item Organize your work, in a reasonably neat and coherent way, in
the space provided. If you wish for something to not be graded, please strike it out neatly. I will grade only work on the exam paper, unless you clearly indicate your desire for me to grade work on additional pages.


\item You needn't spend your time rewriting definitions or axioms on the exam.

\item {\bf Show all your work.  Correct answers without supporting work may not receive credit.}
  
\end{itemize}


\end{minipage}
\hfill
\begin{minipage}[t]{2.3in}
\vspace{0pt}
%\cellwidth{3em}
\gradetablestretch{2}
%Uncomment this line to make the table display 100 as the total no matter what. This is good for tests with an ommit question.
%\settabletotalpoints{100}
\vqword{Problem}
\addpoints % required here by exam.cls, even though questions haven't started yet.	
\gradetable[v]%[pages]  % Use [pages] to have grading table by page instead of question

\end{minipage}

\begin{itemize}

\item When you have completed your test, hand it to me and have a great night.


\end{itemize}

\newpage

\begin{questions}
\addpoints
\question
\begin{parts}
\part[4]  Use the definition of the derivative to calculate the derivative of $f(x) = 3x^2 + 2$
  \vfill

\part[2] Evaluate $f'(2)$
\vfill
\part[4] Determine the equation of the tangent line to $f(x)$ at $x=2$.
\vfill

\part[2] What are the $x$ and $y$ intercepts for the line?

\end{parts}

\newpage
\addpoints

\question Compute the derivatives of the following functions.  Show all of your work.
\begin{parts}
\part[5] $g(x)=2x^2 - \displaystyle\frac{1}{x}$
\vfill
\part[5] $h(x) =\frac{1-4x}{4x-1}$
\vfill
\part[5] $f(x) =(x^3+x+1)(\sqrt{x}-x)$
\vfill
\end{parts}


\newpage 
\addpoints

\question Evaluate the following limits
\begin{parts}
  \part[5] $\lim_{x\rightarrow\infty} \frac{4x^2-1}{x+2}=$
  \vfill
   For $\displaystyle  f(x) = \begin{cases}-x+2 \;\; \text{if}\; x<0 \\ x^2-1 \;\; \text{if}\; x\geq0 \end{cases}$
  \part[5] $\lim_{x\rightarrow -1} f(x)=$
   \vfill
  \part[5] $\lim_{x\rightarrow 0} f(x)=$
  \vfill
\end{parts}

\newpage
\addpoints
\question For the function
$$
\displaystyle  f(x) = \begin{cases}
  x+b \;\;\; \text{if}\; x<1 \\
  x^2+3 \;\; \text{if}\; x\geq1
  \end{cases}
$$
where $b$ is a real number. Evaluate,
\begin{parts}
  \part[2] $\lim_{x\rightarrow 1^-} f(x) =$  
    \vfill
    \part[2] $ \lim_{x\rightarrow 1^+} f(x)=$
  \vfill
  \part[4] Find the value of $b$ such that $f(x)$ is continuous at $x=1$
  \vfill
\end{parts}



\end{questions}
\end{document}
