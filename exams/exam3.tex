\documentclass[12pt,letterpaper]{exam}
\RequirePackage{amssymb, amsfonts, amsmath, latexsym, verbatim, xspace, setspace, mathrsfs}
\usepackage{amsmath,amsthm,amssymb,amsfonts, 
hyperref, color, graphicx, enumitem}
\RequirePackage{tikz, pgflibraryplotmarks}
\usepackage{geometry}
\usepackage{graphicx}
\usepackage{pstricks-add}
\usepackage{auto-pst-pdf}


\newcommand{\N}{\mathbb{N}}
\newcommand{\Z}{\mathbb{Z}}
\newcommand{\Q}{\mathbb{Q}}
 \everymath{\displaystyle}
\newenvironment{problem}[2][Problem:]{\begin{trivlist}
\item[\hskip \labelsep {\bfseries #1}\hskip \labelsep {\bfseries #2}]}{\end{trivlist}}

\newenvironment{claim}[2][Claim:]{\begin{trivlist}
\item[\hskip \labelsep {\bfseries #1}\hskip \labelsep {\bfseries #2}]}{\end{trivlist}}

\newenvironment{defn}[2][Definition:]{\begin{trivlist}
\item[\hskip \labelsep {\bfseries #1}\hskip \labelsep {\bfseries #2}]}{\end{trivlist}}

% Here's where you edit the Class, Exam, Date, etc.
\newcommand{\class}{Calculus 1 for Social Sciences}
\newcommand{\term}{Winter 2019}
\newcommand{\examnum}{}
\newcommand{\examdate}{Exam 3}
\newcommand{\timelimit}{90 min}

% For an exam, single spacing is most appropriate
\singlespacing
% \onehalfspacing
% \doublespacing

% For an exam, we generally want to turn off paragraph indentation
\parindent 0ex

\begin{document} 

% These commands set up the running header on the top of the exam pages
\pagestyle{head}
\firstpageheader{Name:}{Exam 3}{}
\runningheader{\class}{\examnum\ - Page \thepage\ of \numpages}{\examdate}
\runningheadrule

\begin{flushright}
\begin{tabular}{p{2.8in} r l}
\textbf{\class} & \textbf{Name:} & \makebox[2in]{\hrulefill}\\
\textbf{\term} &&\textbf{\examnum}\\
\textbf{\examdate} &&
\textbf{Time Limit: \timelimit}  \\ 
\end{tabular}\\
\end{flushright}
\rule[1ex]{\textwidth}{.1pt}




\begin{minipage}[t]{3.7in}
\vspace{0pt}
\begin{itemize}

\item \textbf{DO NOT open the exam booklet until you are told to begin. You should write your name and section number at the top and read the instructions.}

\vfill

\item Organize your work, in a reasonably neat and coherent way, in
the space provided. If you wish for something to not be graded, please strike it out neatly. I will grade only work on the exam paper, unless you clearly indicate your desire for me to grade work on additional pages.


\item You needn't spend your time rewriting definitions or axioms on the exam.

\item {\bf Show all your work.  Correct answers without supporting work may not receive credit.}
  
\end{itemize}


\end{minipage}
\hfill
\begin{minipage}[t]{2.3in}
\vspace{0pt}
%\cellwidth{3em}
\gradetablestretch{2}
%Uncomment this line to make the table display 100 as the total no matter what. This is good for tests with an ommit question.
%\settabletotalpoints{100}
\vqword{Problem}
\addpoints % required here by exam.cls, even though questions haven't started yet.	
\gradetable[v]%[pages]  % Use [pages] to have grading table by page instead of question

\end{minipage}

\begin{itemize}

\item When you have completed your test, hand it to me and have a great night.


\end{itemize}

\newpage

\begin{questions}
\addpoints
\question For the function $f(x) = \frac{1}{x^4-6x^2}$
\begin{parts}

  \part[2] find the domain of $f(x)$.
  \vfill

  \part[2] what are the $x$ and $y$ intercepts

  \vfill

  \part[4] find the vertical and horizontal asymptotes.
  \vfill
  

  \part[3] Find the critical points of $f(x)$.
  \vfill

  \newpage
  \addpoints

  \part[8] Find the intervals where $f(x)$ is increasing and decreasing.
  \vfill

  \part[3] Find the inflection points of $f(x)$
  \vfill

  \part[8] Find the intervals of concavity.
  \vfill

  
\end{parts}
\newpage
\addpoints
\question[10] Using the information in the previous question plot the graph on the grid below


  \psset{unit=6mm, ticks=none, xlabelsep=1pt, ylabelsep=1pt, arrowinset=0.12}%, l
\psset{gridwidth=0.3pt, subgriddiv=1,gridlabels=0pt}
\begin{pspicture*}(-13,-13)(13,13)
\psgrid(-13,-13)(13,13.5)
\psaxes[labelFontSize=\scriptstyle, linecolor=SteelBlue]{<->}(0,0)(-13,-13)(13,13)[\textsf{X}\rule{0pt}{2.25ex} ,-120][\textsf{Y} , -150]
\psset{linecolor=DodgerBlue4, tickcolor=white, subtickcolor=DodgerBlue4, gridlabelcolor=DodgerBlue4, ,linewidth = 1.2pt}%
\end{pspicture*}

  
\newpage
\question[20]
A farmer with 750 ft of fencing wants to enclose a rectangular area and then devide it into four pens with fencing parallel to one side of the rectangle.  What is the largest possible total area of the four pens?  (be sure to justify that the area is indeed a maximum)




\end{questions}
\end{document}
